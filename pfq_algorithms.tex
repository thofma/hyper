\documentclass[10pt]{article}
\usepackage{amsmath, amssymb, amsthm}
\usepackage{color}
\addtolength{\textwidth}{1.5in}
\addtolength{\hoffset}{-1.0in}
\addtolength{\textheight}{2.0in}
\addtolength{\voffset}{-1.25in}

\numberwithin{equation}{section}
\newtheorem{theorem}{Theorem}[section]
\newtheorem{definition}[theorem]{Definition}
\newtheorem{lemma}[theorem]{Lemma}
\newtheorem{remark}[theorem]{Remark}
\newtheorem{entry}[theorem]{Entry}
\newtheorem{corollary}[theorem]{Corollary}
\newtheorem{proposition}[theorem]{Proposition}
\newtheorem{example}[theorem]{Example}
%\mathtoolsset{showonlyrefs=true}

\newcommand{\F}[5] {{}_{#1}F_{#2} \left( \begin{array}{c} #3 \\ #4 \end{array} \Big| #5  \right)}

\newcommand{\ARG}[2] {\left( \begin{array}{c} #1 \\ #2 \end{array} \right)}

\author{Junxian Li \& Daniel Schultz}
\title{Calculating $\, _{p}F_{q}$}
\date{}

\begin{document}
\maketitle

For $\bold{a}=a_1,\dots,a_p$ and  $\bold{b}=b_1,\dots,b_q$, set $(a)_n := \Gamma(a+n)/\Gamma(a)$ and $(\bold{a})_n := \prod_i (a_i)_n$ and define
\begin{equation}
\label{Fdef}
\F{p}{q}{\bold{a}}{\bold{b}}{z} := \sum_{n=0}^{\infty} \frac{(\bold{a})_n}{(\bold{b})_n} \frac{z^n}{n!}
\end{equation}
The function is undefined when any $b_i$ is $0,-1,-2,\dots$, i.e. $\Gamma(\bold{b}) := \prod_i \Gamma(b_i)$ is infinite. Also, $\hat{\bold{a}}_i$ denotes the length $p-1$ vector with the $i^{\text{th}}$ entry omitted. The standard quantity
$\sigma = \Sigma(\bold{b}) - \Sigma(\bold{a})$ governs several properties of these functions. Whenever a possibly infinite quantity $\Gamma(a_i-a_j)$ appears in a formula, that formula should be interpret via a limiting cases of the general formula.

\section{The case $p < q+1$ [not implemented]}
Just sum the series for any argument. The problem for large $|z|$ is that many terms may be required before the partial sums start to approach the true value. Indeed, in the special case of ${}_{1}F_{1}(a_1;b_1|z)$ the ratio of successive terms is
\begin{equation*}
\frac{a_1+n}{b_1+n} \cdot \frac{z}{n} \approx \frac{z}{n}
\end{equation*}
and at approximately $|z|$ terms have to be summed before the terms start to decrease. In this case, the formal expansion ($n = q+1-p$)
\begin{align*}
\F{p}{q}{\bold{a}}{\bold{b}}{z}&=\sum_{i=1}^{p} \frac{\Gamma(\bold{b}) \Gamma(\hat{\bold{a}}_i-a_i)}{\Gamma(\bold{b}-a_i) \Gamma(\hat{\bold{a}}_i)} (-z)^{-a_i} \F{q+1}{p-1}{a_i,1+a_i-\bold{b}}{1+a_i-\hat{\bold{a}}_i}{\frac{(-1)^n}{z}}\\
&+\sum_{\alpha^n=1} \frac{\Gamma(\bold{b})}{n (2 \pi )^{\frac{n-1}{2}} \Gamma(\bold{a})} e^{n \alpha z^{1/n}} z^{\frac{n-1}{2 n}-\frac{\sigma
   }{n}}\left(1 + \text{ series in } \frac{1}{\alpha z^{1/n}} \right)\text{,}
\end{align*}
which consists of $q+1$ formal series, is useful. The first $p$ series are hypergeometric and $1/n$--Borel summable. The last $n$ series are $?$--Borel summable and the coefficients satisfy recurrences of order ?. TODO: check this.



\section{The case $p > q+1$ [not implemented]}
The formal series is divergent except for zero argument or terminating parameters and is $1/(p-q-1)$--Borel summable in a range of directions. For nonzero arguments this leads to (``The Borel Sum of Divergent Barnes Hypergeometric Series and its Application to a Partial Differential Equation'' by Kunio Ichinobe)
\begin{equation*}
\F{p}{q}{\bold{a}}{\bold{b}}{z} = \sum_{i=1}^{p} \frac{\Gamma(\bold{b}) \Gamma(\hat{\bold{a}}_i-a_i)}{\Gamma(\bold{b}-a_i) \Gamma(\hat{\bold{a}}_i)} (-z)^{-a_i} \F{q+1}{p-1}{a_i,1+a_i-\bold{b}}{1+a_i-\hat{\bold{a}}_i}{\frac{(-1)^{p-q-1}}{z}}
\end{equation*}
The series on the right are convergent.

The difficulty here is when $|z|$ is so small that the convergent series on the right hand side cannot be summed. In this case, a direct evaluation of the Laplace integral defining the Borel sum should be preferred. This proceeds
as follows. For any ``direction'' $\omega$ with $\Re(\omega) > 0$, we have
\begin{equation*}
k! = \int_0^{\infty} e^{-\omega t} (\omega t)^k \omega dt
\end{equation*}
Divide each term of \eqref{Fdef} by $((p-q-1)n)!$ to obtain the convergent series
\begin{equation*}
f(x) := \sum_{n=0}^{\infty} \frac{1}{((p-q-1)n)!}\frac{(\bold{a})_n}{(\bold{b})_n}\text{.} \frac{x^n}{n!}
\end{equation*}
Then, at least formally, with $k = (p-q-1)n$, we have
\begin{equation*}
\F{p}{q}{\bold{a}}{\bold{b}}{z} = \int_{0}^{\infty} e^{-\omega t} f(z(\omega t)^{p-q-1}) \omega dt \text{,}
\end{equation*}
and $\omega$ must be chosen so that the integrand does not hit the singularity of $f(x)$ at $x = (p-q-1)^{p-q-1}$.


\section{The case $p=q+1$}

\subsection{inside unit circle}
For arguments sufficiently inside the unit circle, just sum the series.

\subsection{outside unit circle}
For $z \not \in [0,1]$ (?),
\begin{equation*}
\F{p}{p-1}{\bold{a}}{\bold{b}}{z} = \sum_{i=1}^{p} \frac{\Gamma(\bold{b}) \Gamma(\hat{\bold{a}}_i-a_i)}{\Gamma(\bold{b}-a_i) \Gamma(\hat{\bold{a}}_i)} (-z)^{-a_i} \F{p}{p-1}{a_i,1+a_i-\bold{b}}{1+a_i-\hat{\bold{a}}_i}{\frac{1}{z}}
\end{equation*}
and for arguments sufficiently outside the unit circle we can just sum the series on the right.

\subsection{near unit circle, away from one}
For any argument outside the branch cut $[1,\infty]$, the series on the right hand side of
\begin{equation*}
(1+z)^{-2 a_p} \F{p}{p-1}{\bold{a}}{\bold{b}}{\frac{4 z}{(1+z)^2}} = \sum_{n=0}^{\infty}u_n z^n, \quad |z| < 1
\end{equation*}
is convergent. However, since computation of the $u_n$'s is a bit expensive, it should only be used when absolutely neccessary. There is a good reason for the prefactor $(1+z)^{-2a_p}$: is present in many quadratic transformation formulas in special cases and has the effect of lowering the order of the recurrence relation for $u_n$ by one.

It is also possible to use Pad\'e approximants here, but do we have useful error bounds?

\subsection{near one}

This is the most interesting case as the function can fail to be defined at one. The existence of $F(1)$ is determinded by $\Re(\sigma)> 0$. If $\sigma$ is not an integer we have
\begin{equation}
\label{nearone}
\F{p}{p-1}{\mathbf{a}}{\mathbf{b}}{1-z} = \sum_{n=0}^{\infty} u_n \left(\begin{array}{c} \mathbf{a}\\\mathbf{b}\end{array} \right) z^{\sigma+n} + \sum_{n=0}^{\infty} v_n \left(\begin{array}{c} \mathbf{a}\\\mathbf{b}\end{array} \right) z^n
\end{equation}
with the $u_1,u_2,\dots$ determined from recurrences by $u_0 = \Gamma(-\sigma)\Gamma(\bold{b})/\Gamma(\bold{a})$ and the $v_{p-1}, v_p, \dots$ are determined from recurrences by $v_0, \dots, v_{p-2}$. Thus the difficulty is computing these $v_0, \dots, v_{p-2}$.

If $\sigma$ is an integer, then at most one $\log(z)$ enters into the series.
\subsubsection{generic approach}
We simply evaluate Equation \eqref{nearone} and its derivatives up to and including order $p-1$ at $z=1/4$ to solve for the $u_0,v_0,\dots,v_{p-2}$. The explicit formula for $u_0$ is surprisingly useless in this approach.

\subsubsection{Buehring}
Here we sum the first $m$ terms of Equation \eqref{Fdef} and use a formula derived by Buehring to sum the remaining terms. Since we will generically be dealing with logarithmically convergent series (when $z=1$) in both sums, it is important to balance the choice of $m$ between the two to ensure a sub-exponential algorithm.
We have (Equations (2.7) and (2.9) in ``analytic continuation of the generalized hypergeoemtric sereis near unit argument with emphasis on the zero-balanced series'' by Buehring and Srivastava)
\begin{align}
\nonumber
\sum_{n=m}^{\infty}{\frac{(\bold{a})_n}{(\bold{b})_n} \frac{z^n}{n!}}&=\frac{\Gamma(\bold{b})}{\Gamma(\bold{a})} z^m \sum_{k=0}^{\infty} {\frac{\Gamma(\bold{a}+m+k)}{\Gamma(\bold{b}+m+k)} \frac{z^k}{\Gamma(1+m+k)}}\\
\label{buehring}
&=\frac{\Gamma(\bold{b})(a_p)_m}{\Gamma(\hat{\bold{a}}_p)} z^m \sum_{k=0}^{\infty} {A_k\left(\begin{array}{c} \hat{\bold{a}}_p \\ \bold{b} \end{array}\right)} \,{} _2\tilde{F}_1\left( \begin{array}{c} 1,a_p+m \\ 1+\sigma+a_p+m+k \end{array} \Big| z\right)
\end{align}
where the $A_k(\hat{\bold{a}}_p; \bold{b})$ are independent of $m$ and are polynomials in $a_1,\dots,a_{p-1},b_1,\dots,b_{p-1}$. They can be defined in the base case $p=2$ as
\begin{equation*}
A_k\left(\begin{array}{c} a_1 \\ b_1 \end{array}\right) = \frac{(1-a_1)_k(b_1-a_1)_k}{k!}
\end{equation*}
and inductively for larger $p$ by Hadamard and Cauchy products. After all is said and done, the $A_k$ satisfy an order $p-1$ recurrence and are bounded as
\begin{equation}
\frac{A_k}{k!} \ll \sum_{i<p}k^{\sigma+a_p-1-a_i}
\end{equation}
Now set
\begin{equation*}
F_k = \, _2\tilde{F}_1\left( \begin{array}{c} 1,a_p+m \\ 1+\sigma+a_p+m+k \end{array} \Big| z\right)
\end{equation*}
We have
\begin{equation*}
F_k=\frac{\left(k+\sigma -1 - (1-z) \left(a_p+2 k+m+2 \sigma
   -2\right)\right) F_{k-1} +(1-z) F_{k-2}}{z (k+\sigma )
   \left(a_p+k+m+\sigma -1\right)}
\end{equation*}
and therefore the bound
\begin{equation*}
k!F_k \ll k^{-\sigma} \left|1-1/z\right|^k + k^{-m-\sigma-a_p}
\end{equation*}
To ensure convergence of the tail series, we should have
$|1-1/z|<1$ and $m+\Re(a_i)>0$ for all $i<p$.

In reality the majorant method will probably produce a much worse explicit bound $|A_k/k!| \le c k^{\mu}$ so we are balancing the sum of the first $m$ terms of a sum whose terms are like $n^{-1-\sigma}$ with another series that we can only prove has terms like $k^{\mu-m-\sigma-a_p}$. Any reasonable overestimation of $\mu$ can be compensated by a larger $m$. Finally, in order to sum in total no more than $O(d)$ terms for $d$ digit accuracy, it probably suffices to take $m \approx d$ for reasonable parameter ranges.

\subsubsection{hybrid approach for $\sigma \not \in \mathbb{Z}$}
The necessary coefficients $u_0$ (respectively $v_0,\dots,v_{p-2}$) in \eqref{nearone} may be evaluated by combining \eqref{buehring} (with $z$ replaced by $1-z$) for $m=0$ (respectively large $m$) with the expansion
\begin{equation}
\label{2f1nearone}
\begin{aligned}
{}_2\tilde{F}_1\left(\begin{array}{c} 1,a_p+m \\ 1+\sigma+a_p+m+k \end{array} \Big| 1-z\right) &= \frac{\Gamma (-\sigma-k)}{\Gamma (a_p+m)} \sum_{j=0}^{\infty} \frac{(\sigma+a_p+m+k)_j}{j!}z^{\sigma+k+j}\\
&+\frac{-1}{\Gamma
   \left(\sigma+a_p+m+k\right)} \sum_{j=0}^{\infty} \frac{(a_p+m)_j}{(-\sigma-k)_{j+1}} z^j\text{.}
\end{aligned}
\end{equation}
The basic idea is that
\begin{equation}
\label{viapprox}
\sum_{n=0}^{m-1} \frac{(\bold a)_n}{(\bold b)_n}\frac{(1-z)^n}{n!} \approx v_0 + v_1 z + v_2 z^2 + \cdots\text{.}
\end{equation}
This is no help in evaluating the $u_n$, but those can be found easily. We have

\begin{align*}
F(1-z)=\sum_{n=0}^{m-1} \frac{(\bold a)_n}{(\bold b)_n}\frac{(1-z)^n}{n!}&+\frac{\Gamma(\bold{b})(a_p)_m}{\Gamma(\hat{\bold{a}}_p)} (1-z)^m \sum_{k=0}^{\infty} {A_k\left(\begin{array}{c} \hat{\bold{a}}_p \\ \bold{b} \end{array}\right)}\,{} _2\tilde{F}_1\left( \begin{array}{c} 1,a_p+m \\ 1+\sigma+a_p+m+k \end{array} \Big| 1-z\right)\\
=\sum_{n=0}^{m-1} \frac{(\bold a)_n}{(\bold b)_n}\frac{(1-z)^n}{n!}&+\frac{\Gamma(\bold{b})(a_p)_m}{\Gamma(\hat{\bold{a}}_p)} (1-z)^m \sum_{k=0}^{\infty} {A_k\left(\begin{array}{c} \hat{\bold{a}}_p \\ \bold{b} \end{array}\right)}\frac{\Gamma (-\sigma-k)}{\Gamma (a_p+m)} \sum_{j=0}^{\infty} \frac{(\sigma+a_p+m+k)_j}{j!}z^{\sigma+k+j}\\
&+\frac{\Gamma(\bold{b})(a_p)_m}{\Gamma(\hat{\bold{a}}_p)} (1-z)^m \sum_{k=0}^{\infty} {A_k\left(\begin{array}{c} \hat{\bold{a}}_p \\ \bold{b} \end{array}\right)} \frac{-1}{\Gamma
	\left(\sigma+a_p+m+k\right)} \sum_{j=0}^{\infty} \frac{(a_p+m)_j}{(-\sigma-k)_{j+1}} z^j.
\end{align*}
Taking $m=0$ and equating non-integral powers of $z$ gives
\begin{align*}
\sum_{n=0}^\infty u_n z^{\sigma + n} &= \frac{\Gamma(\bold{b})}{\Gamma(\hat{\bold{a}}_p)} \sum_{k=0}^{\infty} {A_k\left(\begin{array}{c} \hat{\bold{a}}_p \\ \bold{b} \end{array}\right)}\frac{\Gamma (-\sigma-k)}{\Gamma (a_p)} \sum_{j=0}^\infty \frac{(\sigma+a_p+k)_j}{j!} z^{\sigma+k+j}\text{,}
\end{align*}
so that in particular $u_0=\Gamma(-\sigma)\Gamma(\bold b) / \Gamma (\bold a)$ and in general
\begin{align*}
u_n=\frac{\Gamma(\bold{b})}{\Gamma(\hat{\bold{a}}_p)} \sum_{k+j=n}{A_k\left(\begin{array}{c} \hat{\bold{a}}_p \\ \bold{b} \end{array}\right)}\frac{\Gamma (-\sigma-k)}{\Gamma (a_p)} \frac{(\sigma+a_p+k)_j}{j!}\text{.}
\end{align*}
Equating integral powers of $z$ gives 
\begin{align*}
\sum_{n=0}^\infty v_nz^n=\sum_{n=0}^{m-1}{\frac{(\bold{a})_n}{(\bold{b})_n} \frac{(1-z)^n}{n!}}+\frac{\Gamma(\bold{b})(a_p)_m}{\Gamma(\hat{\bold{a}}_p)} \sum_{\ell=0}^\infty\frac{(-m)_\ell}{\ell!}z^\ell \sum_{k=0}^{\infty} {A_k\left(\begin{array}{c} \hat{\bold{a}}_p \\ \bold{b} \end{array}\right)} \frac{-1}{\Gamma
	\left(\sigma+a_p+m+k\right)} \sum_{j=0}^{\infty} \frac{(a_p+m)_j}{(-\sigma-k)_{j+1}} z^j\text{,}
\end{align*}
or, for arbitrary $m$,
\begin{align*}
v_i - [z^i] \sum_{n=0}^{m-1}{\frac{(\bold{a})_n}{(\bold{b})_n} \frac{(1-z)^n}{n!}} &= \sum_{\ell +j=i}\frac{\Gamma(\bold{b})(a_p)_m}{\Gamma(\hat{\bold{a}}_p)}\frac{(-m)_\ell}{\ell!}\sum_{k=0}^{\infty} {A_k\left(\begin{array}{c} \hat{\bold{a}}_p \\ \bold{b} \end{array}\right)} \frac{-1}{\Gamma
	\left(\sigma+a_p+m+k\right)}\frac{(a_p+m)_j}{(-\sigma-k)_{j+1}}\\
&= \frac{\Gamma(\bold{b}) (a_p)_m}{\Gamma(\hat{\bold{a}}_p) \Gamma\left(\sigma+a_p+m\right)} \sum_{j=0}^{i}\frac{(a_p+m)_j (-m)_{i-j}}{(i-j)!}\sum_{k=0}^{\infty} {A_k\left(\begin{array}{c} \hat{\bold{a}}_p \\ \bold{b} \end{array}\right)} \frac{-1}{\left(\sigma+a_p+m\right)_k (-\sigma-k)_{j+1}}
\end{align*}
This formula for the $v_i$ is as effective as the series in the previous section. Another formula for the $v_i$ follows by noting that both sides of \eqref{nearone} can be differentiated, which yields
\begin{equation*}
i! (-1)^i v_i \ARG{\mathbf{a}}{\mathbf{b}} = \frac{(\mathbf{a})_i}{(\mathbf{b})_i} v_0 \ARG{\mathbf{a}+i}{\mathbf{b}+i}\text{.}
\end{equation*}
Combining this with the above formula for $v_0$ gives
\begin{align*}
i! (-1)^i v_i \ARG{\mathbf{a}}{\mathbf{b}} &= \frac{(\mathbf{a})_i}{(\mathbf{b})_i} \left(  \sum_{n=0}^{m-1}{\frac{(\mathbf{a}+i)_n}{(\mathbf{b}+i)_n} \frac{1}{n!}} + \frac{\Gamma(\mathbf{b}+i) \Gamma(a_p+i+m)}{\Gamma(\mathbf{a}+i) \Gamma(\sigma+a_p+m)} \sum_{k=0}^{\infty} A_k \ARG{\hat{\mathbf{a}}_p+i}{\mathbf{b}+i} \frac{1}{(\sigma-i+k) (\sigma+a_p+m)_k} \right)\\
&=\sum_{n=0}^{m-1}{\frac{(\mathbf{a})_{n+i}}{(\mathbf{b})_{n+i}} \frac{1}{n!}} + \frac{\Gamma(\mathbf{b}) \Gamma(a_p+i+m)}{\Gamma(\mathbf{a}) \Gamma(\sigma+a_p+m)} \sum_{k=0}^{\infty} A_k \ARG{\hat{\mathbf{a}}_p+i}{\mathbf{b}+i} \frac{1}{(\sigma-i+k) (\sigma+a_p+m)_k}\text{.}
\end{align*}
Finally, replacing $m$ by $m-i$ gives the succinct formula
\begin{equation}
\label{visuc}
\begin{aligned}
i! (-1)^i v_i \ARG{\mathbf{a}}{\mathbf{b}} &= \quad\frac{\Gamma(\mathbf{b}) \Gamma(a_p+m)}{\Gamma(\mathbf{a}) \Gamma(\sigma-i+a_p+m)} \sum_{k=0}^{\infty} A_k \ARG{\hat{\mathbf{a}}_p+i}{\mathbf{b}+i} \frac{1}{(\sigma-i+k) (\sigma-i+a_p+m)_k}\\
&\quad +\frac{\partial^i}{\partial z^i} \Big|_{z=1} \sum_{n=0}^{m-1}{\frac{(\mathbf{a})_{n}}{(\mathbf{b})_{n}} \frac{z^n}{n!}}\text{.}
\end{aligned}
\end{equation}
When $\Re(\sigma) \le i$ so that the coefficient of $z^i$ on the left hand side of \eqref{viapprox} diverges as $m$ tends to $\infty$, there is cancelation between the two sums on the right hand side of \eqref{visuc}. Thus, it should not be used in this case.

\subsubsection{hybrid approach for $\sigma \in \mathbb{Z}$}
The case of integral $\sigma$ in \eqref{2f1nearone} is in general too messy. Since the final formula is not going to work well for $\sigma < 0$ anyways, let $\sigma$ be a nonnegative integer. In this case \eqref{nearone} takes the shape
\begin{equation*}
\F{p}{p-1}{\mathbf{a}}{\mathbf{b}}{1-z} = \sum_{n=0}^{\infty} w_n \left(\begin{array}{c} \mathbf{a}\\\mathbf{b}\end{array} \right) z^{\sigma+n} \log(z) + \sum_{n=0}^{\infty} r_n \left(\begin{array}{c} \mathbf{a}\\\mathbf{b}\end{array} \right) z^n\text{,}
\end{equation*}
and \eqref{2f1nearone} becomes
\begin{equation*}
\begin{aligned}
{}_2\tilde{F}_1\left(\begin{array}{c} 1,a_p+m \\ 1+\sigma+a_p+m+k \end{array} \Big| 1-z\right) &= \sum_{j=0}^{\sigma} \frac{(-1)^j z^j}{\Gamma(a_p+m)} \begin{cases} \frac{\Gamma (-j+k+\sigma ) \Gamma
   \left(j+m+a_p\right)}{\Gamma (k+\sigma +1) \Gamma
   \left(k+m+\sigma +a_p\right)} \text{,} \quad j < \sigma + k \\
 \frac{-\psi ^{(0)}\left(m+\sigma+a_p\right)-\log(z)-\gamma }{\sigma !} \text{,} \quad j = \sigma+k \end{cases}\\
&+O(z^{\sigma+1})\text{.}
\end{aligned}
\end{equation*}
Thus the coefficients of $z^0,z^1,\dots,z^{\sigma}$ in $F(1-z)$ as polynomials in $\log(z)$ are equal to those of
\begin{equation*}
\begin{gathered}
\sum_{n=0}^{m-1}\frac{(\mathbf{a})_n}{(\mathbf{b})_n} \frac{(1-z)^n}{n!}+ \frac{\Gamma(\bold{b})(a_p)_m}{\Gamma(\hat{\bold{a}}_p)} (1-z)^m \left(\,{} _2\tilde{F}_1\left( \begin{array}{c} 1,a_p+m \\ 1+\sigma+a_p+m+0 \end{array} \Big| 1-z\right) \right.\\
+ \left. \sum_{k=1}^{\infty} {A_k\left(\begin{array}{c} \hat{\bold{a}}_p \\ \bold{b} \end{array}\right)} \,{} _2\tilde{F}_1\left( \begin{array}{c} 1,a_p+m \\ 1+\sigma+a_p+m+k \end{array} \Big| 1-z\right) \right)
\end{gathered}
\end{equation*}
or
\begin{equation*}
\begin{gathered}
\sum_{n=0}^{m-1}\frac{(\mathbf{a})_n}{(\mathbf{b})_n} \frac{(1-z)^n}{n!}+ \frac{\Gamma(\bold{b})(a_p)_m}{\Gamma(\hat{\bold{a}}_p)} (1-z)^m \left(\sum_{j=0}^{\sigma-1} z^j\frac{-(m+a_p)_j}{(-\sigma-0)_{j+1} \Gamma
   \left(0+m+\sigma +a_p\right)} \right.\\
+ \left. \frac{(-1)^\sigma z^\sigma}{\Gamma(a_p+m)} \frac{-\psi \left(m+\sigma+a_p\right)-\gamma-\log(z) }{\sigma !} \right.\\
+ \left. \sum_{k=1}^{\infty} {A_k\left(\begin{array}{c} \hat{\bold{a}}_p \\ \bold{b} \end{array}\right)} \sum_{j=0}^{\sigma} z^j\frac{-(m+a_p)_j}{(-\sigma-k)_{j+1} \Gamma
   \left(k+m+\sigma +a_p\right)} \right)
\end{gathered}
\end{equation*}




\section{implementation}

\subsection{Tight ${}_2 F_1$ bounds everywhere}
The analysis is for real parameters $a,b,c \in \mathbb{R}$, but it should be possible to do something for complex parameters too.

With
\begin{equation}
f(w) = (1+w)^{-2 a} \, _2F_1\left(\begin{array}{c} a,b \\ c \end{array} \Big| \frac{4 w}{(1+w)^2}\right) = \sum_{n=0}^{\infty}r_n w^n, \quad |w| < 1
\end{equation}
we have $r_0=1$, $r_1=\frac{4ab}{c} - 2a$, and $r_{n+1} = \lambda_0(n) r_n + (1-\lambda_1(n)) r_{n-1}$, where
\begin{align*}
\lambda_0(n) &= \frac{2 (2 b-c) (n+a)}{(n+1) (n+c)}\\
\lambda_1(n) &= \frac{2 (1-2 a+c) (n+a)}{(n+1) (n+c)}
\end{align*}
The unit disk $|w| < 1$ is mapped into the whole complex $z$-plane minus $[1,\infty)$ by $z=\frac{4w}{(1+w)^2}$, hence this provides a method for computing the usual branch of $\, _2F_1$ if we can bound the tails of the sum. Note that $\lambda_0, \lambda_1 \to 0$, and for the moment entertain the assumption that $|\lambda_0| \le \lambda_1 \le 1$ for all $n$:
\begin{align*}
|r_2| &= |\lambda_0 r_1 + (1 - \lambda_1) r_0| \\
& \le |\lambda_0||r_1| + (1 - \lambda_1) |r_0| \\
& \le  (|\lambda_0| + 1 - \lambda_1) \operatorname{max}(|r_0|, |r_1|) \\
& \le \operatorname{max}(|r_0|, |r_1|)\text{.}
\end{align*}
Hence $|r_n| \le \operatorname{max}(|r_0|, |r_1|)$ for all $n$ by induction. For general real parameters $a, b, c$ the inequality $|\lambda_0(n)| \le \lambda_1(n)$ is not possible for all $n$ as singularities (either logarithmetic or algebraic) of the $\, _2F_1$ at $z=\infty$ and $z=1$ mean that the $r_n$ can grow like an arbitrarily large power of $n$.

To remedy this, consider $\tilde{r}_n := r_n n^{-\mu}$ for some arbitrary real $\mu$. The transformed recurrence is $\tilde{r}_n = \tilde{\lambda}_0(n) \tilde{r}_{n-1} + (1 - \tilde{\lambda}_1(n)) \tilde{r}_{n-2}$ where
\begin{align*}
\tilde{\lambda}_0(n) &= \left(\frac{n}{n+1}\right)^{\mu} \lambda_0(n)\\
\tilde{\lambda}_1(n) &= 1 - \left(\frac{n-1}{n+1}\right)^{\mu} (1-\lambda_1(n))
\end{align*}
If $|\tilde{\lambda}_0(n)| \le \tilde{\lambda}_1(n) \le 1$ for all $n \ge n_0$, then it follows as above that $r_n \le \max(|\tilde{r}_{n_0}|, |\tilde{r}_{n_0-1}|) n^{\mu}$ for all $n > n_0$. There are two ways to turn this into an algorithm for bounding the tails. Either choose an $n_0$ and compute a $\mu$ (not recommended), or since
\begin{align*}
\tilde{\lambda}_0(n) &= 2 (2 b-c) n^{-1}+O\left(n^{-2}\right)\\
\tilde{\lambda}_1(n) &= 2 (1-2 a+c+\mu) n^{-1} + O\left(n^{-2}\right)
\end{align*}
we can choose any $\mu >-1 + 2 a - c + |2 b-c|$ and compute an $n_0$. This is an optimal bound on $\mu$.

\subsection{Tight ${}_3 F_2$ bounds near $1$}

Series expansions of solutions around $z=1$ can be constructed as
\begin{equation*}
\sum_{n=0}^{\infty} r_n (1-z)^{n+\lambda}
\end{equation*}
where $\lambda =0$ or $\lambda = b_1+b_2-a_1-a_2-a_3$ and $r_{n+2} + \kappa_1(n) r_{n+1} + \kappa_0(n) r_n = 0$ where
\begin{align*}
\kappa_0(n) &= \frac{\left(a_1+\lambda +n\right) \left(a_2+\lambda
   +n\right) \left(a_3+\lambda +n\right)}{(\lambda +n+1)
   (\lambda +n+2) \left(a_1+a_2+a_3-b_1-b_2+\lambda
   +n+2\right)}\\
&=1 + \left(b_1+b_2-5 \right) n^{-1} + O(n^{-2})\\
\kappa_1(n) &=-2 - \left(b_1+b_2-5 \right) n^{-1} + O(n^{-2})
\end{align*}
For $\lambda = b_1+b_2-a_1-a_2-a_3$ the $r_n$ are determined once $r_0$ is fixed, while for $\lambda = 0$, the $r_n$ depend freely on $r_0$ and $r_1$. This gives $3$ solutions.

By the substitution $r_n = \tilde{r}_n n^\mu$ where $\mu = -2 + \operatorname{max}(b_1,b_2)$, this equation can be brought to the form
\begin{equation*}
\tilde{r}_{n+2} + \left(-2 + \frac{d_1}{n} + \frac{d_2}{n^2} + O(\frac{1}{n^3}) \right) \tilde{r}_{n+1} + \left(1 - \frac{d_1}{n} - \frac{d_2}{n^2} + O(\frac{1}{n^3}) \right) \tilde{r}_{n} = 0
\end{equation*}
where crutially $d_1 = 1 + |b_1-b_2|$ is positive. This equation can be rewritten as
\begin{equation*}
\tilde{r}_{n+2} - \tilde{r}_{n+1} = \left(1 - \frac{d_1}{n} - \frac{d_2}{n^2} \right) (\tilde{r}_{n+1} - \tilde{r}_{n}) + O(\frac{\operatorname{max}(|\tilde{r}_{n+1}|, |\tilde{r}_{n}|)}{n^3})
\end{equation*}
All constants hidden by the O notation are effective and depend only on the parameters $b_i,a_i$. We would like to show that $\tilde{r}_{n} = O(n^{\epsilon})$ for every $\epsilon > 0$.


\subsection{majorant method}

This is a terse summary of Messarobba. We would like to study the various functions
\begin{equation*}
F\left(z\right)\text{,} \quad F\left(\frac{1}{z}\right)\text{,} \quad F\left(1-z\right)\text{,} \quad  (1+z)^{-2a_1} F\left(\frac{4z}{(1+z)^2}\right)\text{,} \quad \dots
\end{equation*}
as convergent power series for $|z|<1$ as this allows for the computation of $F$ everywhere. In order to evaluate these power series, we need bounds on the coefficients, and tight bounds are already difficult to prove for ${}_2 F_1$ and ${}_3 F_2$. If we are not near the radius of convergence of these series, an overestimation of the coefficients is acceptable if it allows us to actually get proven bounds.

Each of these functions $f(z)$ satisfies a homogeneous linear differential equation $P(f(z))=0$ which will we write in terms of $\theta = z \frac{d}{dz}$. Since $z\theta = \theta z - z$, we can write the operator $P$ with $\theta$ \emph{on the left}. When $\theta$ is on the left and $z$ is on the right, it is easy to transform the differential equation to a recurrsion on the coefficients. For example, for $F(z) = \, _2F_1\left(\begin{array}{c} a_1,a_2 \\ b_1 \end{array} \Big| z\right) = \sum_{n=0}^{\infty}{u_n} z^n$, we have

\begin{equation*}
P = (\theta+b_1-1)(\theta) - (\theta+a_1-1)(\theta+a_2-1)z \Leftrightarrow \frac{u_n}{u_{n-1}} = \frac{(n+a_1-1)(n+a_2-1)}{(n+b_1-1)(n)}
\end{equation*}

\subsubsection{coefficient recursions}
Write the differential operator as $P(z,\theta) = \theta^r p_r(z) + \dots + \theta p_1(z) + p_0(z) = P_s(\theta) z^s + \dots + P_1(\theta) z + P_0(\theta) \in\mathbb{F}[z,\theta]$ with $\theta$ on the left and assume that $P_0(0) \neq 0$. Define the operator $L(z,\theta) = P(z,\theta) p_r(z)^{-1} = \sum_{j=0}^{\infty} Q_j(\theta) z^j$ and note that $\deg(Q_0(\theta)) = r$ and $\deg(Q_j(\theta)) < r$ for $j > 0$. Let $\lambda \in \overline{\mathbb{F}}$ denote a fixed root of $Q_0$ such that none of $\lambda-1, \lambda-2, \dots$ is a root of $Q_0$. Let $\mu(\nu)$ denote the multiplicity of $\nu$ as a root of $Q_0$ (or as a root of $P_0$). For a double sequence $\{u_{\lambda + n, k}\}_{n,k \ge 0}$, let
\begin{equation*}
u(z) = \sum_{\substack{n=0 \\ \nu=\lambda+n}}^{\infty} \sum_{k=0}^{\infty} u_{\nu,k} z^{\nu} \frac{\log^k z}{k!}\text{,}
\end{equation*}
be a solution to $P(z,\theta)(u(z)) = 0$. This is actually a polynomial in $\log z$, so let $\tau(n)$ be a nondecreasing integer-valued function of $n$ satisfying 
$u_{\lambda+n, k} = 0$ for $k \ge \tau(n)$. We will see shortly that we can take $\tau(0) \le \mu(\lambda + 0)$ and $\tau(n) \le \tau(n-1)+\mu(\lambda+n)$. In terms of the operator $S_k$, which shifts a sequence $\{a_k\}_{k \ge 0}$ to $\{a_{k+1}\}_{k \ge 0}$, the differential equation says that
\begin{equation*}
P_0(\nu + S_k) u_{\nu} = - \sum_{j=1}^{s} P_j(\nu + S_k) u_{\nu - j}
\end{equation*}
Since $P_0(\nu + S_k) = S_k^{\mu(\nu)}( c_0 + c_1 S_k + \cdots )$, this equation allows us to determine all $u_{\lambda+n, k}$ with $k \ge \mu(\lambda+n)$ once the initial values $E_{\lambda} = \{u_{\lambda+n, k} \, | \, 0 \le k < \mu(\lambda+n)\}$ are determined. Considering all possible $\lambda$ gives $r$ linearly independent solutions to $P=0$.

\subsubsection{tail bounds}
Let $K < \tau(\infty)$ denote the higest power of $\log z$ occuring in $u(z)$, and consider the truncation
\begin{equation*}
\tilde{u}(z) = \sum_{n=0}^{N-1} \sum_{k=0}^{K} u_{\lambda + n,k} z^{\lambda + n} \frac{\log^k z}{k!}\text{,}
\end{equation*}
and the normalized residual $q(z)$ defined by $P(z,\theta)(\tilde{u}(z)) = Q_0(\theta) q(z)$. This has the form
\begin{equation*}
q(z) = \sum_{j=0}^{s-1} \sum_{k = 0}^{K} q_{\lambda + N + j, k} z^{\lambda + N + j} \frac{\log^k z}{k!}
\end{equation*}
where the $q_{\lambda+N}, \dots, q_{\lambda+N+s-1}$ can be computed from $P(z,\theta)$ and $u_{\lambda+N-1}, \dots, u_{\lambda+N-s}$.

Consider
$y(z) = p_r(z)(\tilde{u}(z) - u(z))$ as a solution of $L(z, \theta)(y(z)) = Q_0(\theta)(q(z))$. Suppose that for some $n_0 > 0$ we have constructed power series $\hat{a}(z) = \sum_{j>0} \hat{a}_j z^j$, $\hat{q}(z)  = \sum_{n>0} \hat{q}_n z^n$, and $\hat{y}(z) = \sum_{n \ge 0} \hat{y}_n z^n$ with nonnegative coefficients satisfying
\begin{enumerate}
\item For all $j > 0$ and $n \ge n_0$,
\begin{equation*}
n \sum_{t=0}^{\tau(n) - 1} \left| [X^t] \frac{Q_j(\lambda+n+X)}{X^{-\mu(\lambda+n)}Q_0(\lambda+n+X)} \right| \le  \hat{a}_j \text{.}
\end{equation*}
\item For all $n \ge n_0$ and $k \ge 0$, $| q_{\lambda+n,k}| \le \hat{q}_n$.
\item $|y_{\lambda+n,k}| \le \hat{y}_n$ for all $n < n_0$ and $k \ge 0$.
\item $|y_{\lambda+n,k}| \le \hat{y}_n$ for all $n \ge n_0$ and $k < \mu(\lambda + n)$.
\item $\hat{y}(z)$ satsifies
\begin{equation*}
z \hat{y}'(z) = \hat{a}(z) \hat{y}(z) + \hat{q}(z)\text{.}
\end{equation*}
\end{enumerate}
If all of these are true, we have $|z^{-\lambda} y(z)| \le \hat{y}(z)$. The reason for dividing the differential equation by $p_r(z)$ on the right is that $\deg Q_j < \deg Q_0$, so we can expect finite values for the $\hat{a}_j$.

Now, we have
\begin{equation*}
\sum_{j=1}^{\infty} Q_j(\theta) z^j = \frac{P(z,\theta)}{p_r(z)} - Q_0(\theta)
= \frac{P(z,\theta)}{p_r(z)} - \frac{P(0,\theta)}{p_r(0)} \text{.}
\end{equation*}
For all differential equations arising from hypergeometric functions considered here, $\sum_{j=1}^{\infty} Q_j(\theta) z^j$ will be a finite linear combination of functions of the form ($i, k \ge 0$)
\begin{gather*}
z^i, \quad z \frac{\partial}{\partial z} \frac{z^i}{(1-z)^k}, \quad
z \frac{\partial}{\partial z} \log \left( \frac{1}{1-z} \right),\\
z \frac{\partial}{\partial z} \frac{z^i}{(1-z^2)^k}, \quad
z \frac{\partial}{\partial z} \log \left( \frac{1}{1-z^2} \right), \quad
z \frac{\partial}{\partial z} \log \left( \frac{1+z}{1-z} \right),
\end{gather*}
all with nonnegative coefficients as power series in $z$.

\begin{remark}
This is not accurate for equations arising from Borel resummation, where the list needs to be augmented by ?.
\end{remark}

The coefficients of the linear combination, say $f_j(\theta)$, will be polynomials in $\theta$. Bounding the combinations
\begin{equation*}
n \sum_{t=0}^{\tau(n) - 1} \left| [X^t] \frac{f_j(\lambda+n+X)}{X^{-\mu(\lambda+n)} Q_0(\lambda+n+X)} \right|
\end{equation*}
for each $j$ and for all $n \ge n_0$ will give a valid $\hat{a}(z)$ and a nice formula for $\hat{h}(z) = \exp \int_0^z \hat{a}(z)/z dz$. It now suffices to choose a $\hat{q}(z)$ so that
\begin{equation*}
\hat{y}(z)=\hat{h}(z) \int_0^z \frac{\hat{q}(z)/z}{\hat{h}(z)}{dz}
\end{equation*}
satisfies conditions 2 and 4.

\subsection{series evaluation}

For large enough $\tau$, the solution takes the form
\begin{equation*}
f(z) = \sum_{i=0}^{\infty} \sum_{j=0}^{\tau-1} u_{i,j} z^{\lambda+i} \log(z)^j/j!\text{,}
\end{equation*}
and the coefficients $u_{i,j}$ satsify $u_{i_,j} = 0$ for $j \ge \tau$. Therefore, write $u_i = \sum_{j=0}^{\tau-1} u_{i,j} \Lambda^{\tau-1-j}$ where \emph{everything is modulo $\Lambda^\tau$}.
Eventually the coefficients $u_{i}$ satisfy a relation of the form
\begin{equation}
\label{unrec}
u_n = a_1 u_{n-1} + \cdots + a_s u_{n-s}\text{,} \quad a_i \in \mathbb{F}(n)[\Lambda]
\end{equation}
Let $M_n \in \mathbb{F}(n)[\Lambda] ^{s \times s}$ be the companion matrix (with the $a_i$ on the first row) such that
\begin{equation*}
\left(\begin{array}{c}
u_n\\
\vdots \\
u_{n-(s-1)}
\end{array}\right)
=M_n
\left(\begin{array}{c}
u_{n-1}\\
\vdots \\
u_{n-s}
\end{array}\right)
\end{equation*}
Set $f_{[N_0,N_1)}(z) = \sum_{i=N_0}^{N_1-1} z^{\lambda+i} u_i(\log(z))$ where $u_i(\log(z))$ denotes $u_i \in \mathbb{F}[\Lambda]$ with $\Lambda^{\tau-1-j}$ replaced by $\log(z)^j/j!$.
For the derivative $f^{(d)}(z)$ of order $d$ we have
\begin{equation*}
\left(\begin{array}{c}
f_{[N_0,N_1)}^{(d)}(z) \\
? \\
\vdots
\end{array}\right) = \sum_{i=N_0}^{N_1-1} z^{\lambda+i-d} (\Lambda+\lambda+i)^{(d)}\prod_{N_0\le \ell \le i}M_{\ell}
\left(\begin{array}{c}
u_{N_0-1}\\
\vdots \\
u_{N_0-s}
\end{array}\right) (\log(z))\end{equation*}
where $x^{(d)} := x(x-1)\cdots(x-(d-1))$ on the right hand side denotes the falling factorial. Therefore, to evaluate several derivatives of $f$, it suffices to take the first entry of the right hand side for several values of $d$, where the products $\prod_{N_0\le \ell < i}M_{\ell}$ can be reused. Furthermore, the final product $\prod_{N_0\le \ell < N_1}M(\ell)$, when multiplied by the initial values $u_{N_0-1}, \dots u_{N_0-s}$, gives the final $u_{N_1-1}, \dots, u_{N_1-s}$, which are needed for the estimation of the tail $\sum_{i=N_1}^{\infty} z^i u_i(\log(z))$.

To avoid either a catastrophic linear loss of precision when the $a_i$ are approximate quantities or a slow algorithm when the $a_i$ are ``small'' exact quantities, the above sum should be evaluated via binary splitting: that is, for example
\begin{align*}
\sum_{i=0}^{7} z^{i} \prod_{0\le \ell \le i}M_{\ell} &= (M_0 + z M_1 M_0 + z^2(M_2 + z M_3 M_2) M_1 M_0)\\
& \quad + z^4 (M_4 + z M_5 M_4 + z^2(M_6 + z M_7 M_6) M_5 M_4)M_3 M_2 M_1 M_0\text{.}
\end{align*}

\subsection{putting everything together}
This section discusses the reliable evaluation of the solution and its derivatives $f(z), f'(z), \dots, f^{(\delta-1)}(z)$, which can be writen as
\begin{equation*}
f^{(d)}(z) = f_{[0,N_0)}^{(d)}(z) + f_{[N_0,N)}^{(d)}(z) + f_{[N,\infty)}^{(d)}(z)
\end{equation*}
where 
\begin{equation*}
f_{[N_0,N_1)}(z) = \sum_{N_0 \le i < N_1,j} u_{i,j} z^{\lambda+i} \log(z)^j/j!
\end{equation*}
The the quantities $z^{\lambda+i} \log(z)^j/j!$ for integers $i$ and $j$ need to be evaluated reliably. This is a problem when $z$ is zero or a ball containing zero.

The first block $f_{[0,N_0)}^{(d)}(z)$ includes those ``problematic'' terms $u_{i,j} z^{\lambda+i} \log(z)^j/j!$ where $\lambda+i$ is a root of $Q_0(\theta)$. These terms are problematic because the denominators of \eqref{unrec} could vanish, thus they should be dealt with separately. For each of these terms we just evaluate $z^{\lambda+i} \log(z)^j/j!$ directly and take care when $z$ contains zero, where the sign of $\Re(\lambda+i)$ is relevent.

The next block also requires care with respect to the evaluation of $\log(z)$. What we actually get out of the previous section is a reliable evaluation of
\begin{equation*}
z^{d - N_0 - \lambda} f_{[N_0,N)}^{(d)}(z) = \sum_{j=0}^{\tau-1} e_j \Lambda^{\tau-1-j} \in \mathbb{C}[\Lambda]\text{,}
\end{equation*}
that is, we still have to evaluate $\sum_{j=0}^{\tau-1} e_j z^{\lambda+N_0-d}\log(z)^j/j!$ as reliably as possible. For this, it is helpful if $\Re(\lambda+N_0-d) > 0$ which is why it is a good ideal to at least choose an $N_0 \ge \delta$.

For the final block, the majorant method produces a power series $\hat{B}(z) = \sum_{i=0}^{\infty} b_i z^i \in \mathbb{R}_{\ge 0}[z]$ with $z^N \hat{B}(z)$ majorizing the tail $\sum_{i = N}^{\infty} u_{i,j} z^i$ for all $j<\tau$. Thus to bound $|f(z)| + |f'(z)| \epsilon + \cdots + |f^{\delta-1}(z)|/(\delta-1)! \epsilon^{\delta-1}$, we need to calculate, while working in $\epsilon$ modulo $\epsilon^{\delta}$, a majorant (in $\epsilon$) of $(z+\epsilon)^{\lambda+N} \log(z+\epsilon)^j/j!$ for each $j<\tau$, add these up, and multiply the sum by $B(z+\epsilon)$. Since the derivatives of $z^\delta \log(z)^{j}$ up to and including order $\delta-1$ are continuous at $z=0$, it suffices to steal $z^\delta$ from the $z^{\lambda+N}$. If the deficit $z^{\lambda+N-\delta}$ is not continuous at $z=0$, the situation is hopeless anyways.
For fixed $\delta$ we have
\begin{equation*}
(z+\epsilon)^\delta \log(z+\epsilon)^{j} = \sum_{k=0}^{\delta-1} c_{j,k}(\log(z)) z^{\delta-k} \epsilon^{k} + O(\epsilon^\delta)
\end{equation*}
for certain polynomials $c_{j,k}$ of degree $j$ satisfying $c_{j+1,k} = \log(z) c_{j,k} + \sum_{\ell=1}^{k} (-1)^{\ell-1}c_{k-\ell}/\ell$.

\end{document}